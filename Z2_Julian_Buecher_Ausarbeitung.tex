% This is samplepaper.tex, a sample chapter demonstrating the
% LLNCS macro package for Springer Computer Science proceedings;
% Version 2.21 of 2022/01/12
%
%% arara directives
% arara: pdflatex
% arara: biber
% arara: pdflatex
\documentclass[
runningheads,
%draft,
]{llncs}

%----------------------------------------------------------------------------------------
%	PACKAGES AND RELATED DOCUMENT CONFIGURATIONS
%----------------------------------------------------------------------------------------

% T1 fonts will be used to generate the final print and online PDFs, so please use T1 fonts in your manuscript whenever possible.
% Other font encondings may result in incorrect characters.
\usepackage[T1]{fontenc}
\usepackage[ngerman]{babel}

% Used for displaying a sample figure. If possible, figure files should be included in EPS format.
\usepackage{graphicx}
\graphicspath{ {./sections/} }
	
% Reference configuration
\usepackage[
backend=biber, % using biber, bibtex task has to be configured to use biber (Texmaker: Options)
style=ieee, % styles overview: https://de.overleaf.com/learn/latex/Biblatex_bibliography_styles
sorting=ynt, % sorting options: https://de.overleaf.com/learn/latex/Articles/Getting_started_with_BibLaTeX
isbn=false,
url=false,
eprint=false,
related=false,
doi=false
]{biblatex}
\addbibresource{references.bib} 

% If you use the hyperref package, please uncomment the following two lines
% to display URLs in blue roman font according to Springer's eBook style:
%\usepackage{color}
%\renewcommand\UrlFont{\color{blue}\rmfamily}

% other packages
\usepackage{import}
% Abkuerzungsverzeichnis
\usepackage[nohyperlinks, printonlyused, nolist]{acronym}
\usepackage{nameref}
\usepackage{hyperref}
\usepackage{float}
 \usepackage{microtype}

%----------------------------------------------------------------------------------------
% Additional Commands
\newcommand*{\fullref}[1]{\hyperref[{#1}]{\ref*{#1} \nameref*{#1}}} % One single link

%----------------------------------------------------------------------------------------

\setcounter{secnumdepth}{3}

\begin{document}

%----------------------------------------------------------------------------------------
%	DOCUMENT TITLE 
%----------------------------------------------------------------------------------------

\title{Datenschutz in IoT-Systemen \\und deren Anwendungen}
%
%\titlerunning{Abbreviated paper title}
% If the paper title is too long for the running head, you can set
% an abbreviated paper title here
%
\author{Julian Bücher\inst{1}}
%
% First names are abbreviated in the running head.
% If there are more than two authors, 'et al.' is used.
%
\institute{Hochschule Karlsruhe, \\Moltkestraße 30, 76133 Karlsruhe, Deutschland\\
\email{buju1023@h-ka.de}\\
\url{https://www.h-ka.de/infm}}
%
\maketitle              % typeset the header of the contribution

%----------------------------------------------------------------------------------------
%	ABSTRACT AND KEYWORDS
%----------------------------------------------------------------------------------------

\begin{abstract}
Die Nutzung von Geräten des \ac{iot}, im Speziellen der Geräteklasse der Sensoren, ermöglicht es, Informationen aus deren Umgebung zu sammeln, die wiederum übergeordneten Applikationen für die Bereitstellung von Diensten im Rahmen von Handlungsempfehlungen zur Verfügung gestellt werden können.
Hierbei kann es sich einerseits um rein technische Daten, wie zum Beispiel Metriken hochmoderner Fertigungsstraßen, oder auch um personenbezogene Informationen aus der Interaktion von menschlichen Nutzern mit den Geräten innerhalb des privaten oder geschäftlichen Kontextes handeln. 
Auf Basis der aktuellen Rechtssprechung resultieren hieraus besonders schwerwiegende Eingriffe in die Privatsphäre der Nutzer, die bezüglich der Erhebung von persönlichen Informationen, zu jeder Zeit die Möglichkeit besitzen sollten, den Umfang und die Art der Daten zu kontrollieren, die über sie gesammelt werden.
Ziel dieser Arbeit ist es, basierend auf den derzeit geltenden gesetzlichen Regelungen innerhalb des europäischen Raumes zu evaluieren, in welchem Rahmen persönliche Informationen von IoT-Geräten (unkontrolliert) gesammelt werden.
Speziell werden aufgrund der höheren Interaktion von Gerät und Nutzer im Bereich von Smart Home und Smart City, ausschließlich Szenarien in diesem Zusammenhang beleuchtet.

\keywords{Internet of Things  \and Datenschutz \and Privatsphäre \and GDPR \and Smart Home \and Smart City.}
\end{abstract}

%----------------------------------------------------------------------------------------
%	DOCUMENT BODY
%----------------------------------------------------------------------------------------

%\newpage

\import{shortages/}{shortages.tex}

\newpage

\import{sections/introduction/}{introduction.tex}
\import{sections/fundamentals/}{fundamentals.tex}
\import{sections/main/}{body.tex}
\import{sections/main/countermeasures/}{countermeasures.tex}
\import{sections/main/discussion/}{discussion.tex}
\import{sections/ending/}{ending.tex}

%----------------------------------------------------------------------------------------
%	BIBLIOGRAPHY
%----------------------------------------------------------------------------------------

%\newpage

\printbibliography

%----------------------------------------------------------------------------------------

\end{document}
