% This is samplepaper.tex, a sample chapter demonstrating the
% LLNCS macro package for Springer Computer Science proceedings;
% Version 2.21 of 2022/01/12
%
%% arara directives
% arara: pdflatex
% arara: biber
% arara: pdflatex
\documentclass[runningheads]{llncs}

%----------------------------------------------------------------------------------------
%	PACKAGES AND RELATED DOCUMENT CONFIGURATIONS
%----------------------------------------------------------------------------------------

% T1 fonts will be used to generate the final print and online PDFs, so please use T1 fonts in your manuscript whenever possible.
% Other font encondings may result in incorrect characters.
\usepackage[T1]{fontenc}

% Used for displaying a sample figure. If possible, figure files should be included in EPS format.
\usepackage{graphicx} 
	
% Reference configuration
\usepackage[
backend=biber, % using biber, bibtex task has to be configured to use biber (Texmaker: Options)
style=ieee, % styles overview: https://de.overleaf.com/learn/latex/Biblatex_bibliography_styles
sorting=ynt % sorting options: https://de.overleaf.com/learn/latex/Articles/Getting_started_with_BibLaTeX
]{biblatex}
\addbibresource{references.bib} 

% If you use the hyperref package, please uncomment the following two lines
% to display URLs in blue roman font according to Springer's eBook style:
%\usepackage{color}
%\renewcommand\UrlFont{\color{blue}\rmfamily}

% other packages
\usepackage{import}
% Abkuerzungsverzeichnis
\usepackage[printonlyused]{acronym}
\usepackage{nameref}
\usepackage{hyperref}

%----------------------------------------------------------------------------------------
% Additional Commands
\newcommand*{\fullref}[1]{\hyperref[{#1}]{\ref*{#1} \nameref*{#1}}} % One single link

%----------------------------------------------------------------------------------------

\begin{document}

%----------------------------------------------------------------------------------------
%	DOCUMENT TITLE 
%----------------------------------------------------------------------------------------

\title{Datenschutz in IoT-Systemen und deren Anwendungen}
%
%\titlerunning{Abbreviated paper title}
% If the paper title is too long for the running head, you can set
% an abbreviated paper title here
%
\author{Julian Bücher\inst{1}}
%
% First names are abbreviated in the running head.
% If there are more than two authors, 'et al.' is used.
%
\institute{Hochschule Karlsruhe, \\Moltkestraße 30, 76133 Karlsruhe, Deutschland\\
\email{buju1023@h-ka.de}\\
\url{https://www.h-ka.de/infm}}
%
\maketitle              % typeset the header of the contribution

%----------------------------------------------------------------------------------------
%	ABSTRACT AND KEYWORDS
%----------------------------------------------------------------------------------------

\begin{abstract}
Im Rahmen ihrer Funktion sammeln Geräte und Anwendungen innerhalb des Internet of Things (IoT) beständig Daten ihrer Umgebung. Hierbei kann es sich um Daten von äußeren Gegebenheiten, wie z.B. ihrer direkten Umgebung, oder auch um personenbezogene Daten handeln. Hieraus ergeben sich jedoch starke Eingriffe in die Selbstbestimmung der Nutzer, die zu jeder Zeit die Möglichkeit besitzen sollten, den Umfang und die Art der Daten zu kontrollieren, die über sie gesammelt werden. Teil dieser Ausarbeitung ist die Analyse der gesammelten Daten, die im Bereich Smart Home und Smart City von den Nutzern gesammelt wird und deren Einhaltung von rechtlichen Rahmenbedindngungen, wie z.B. der Datenschutz Grundverordnung.

\keywords{Internet of Things  \and Datenschutz \and Privatsphäre \and GDPR \and Smart Home \and Smart City.}
\end{abstract}

%----------------------------------------------------------------------------------------
%	DOCUMENT BODY
%----------------------------------------------------------------------------------------

\newpage

\import{shortages/}{shortages.tex}

\newpage

\import{sections/introduction/}{introduction.tex}

%----------------------------------------------------------------------------------------
%	BIBLIOGRAPHY
%----------------------------------------------------------------------------------------

\newpage

\printbibliography

%----------------------------------------------------------------------------------------

\end{document}
