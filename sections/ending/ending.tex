%% ending.tex
%%

%------------------------------------------------------------------------------------------------
\section{Zusammenfassung}
\label{sec:Zusammenfassung}
%------------------------------------------------------------------------------------------------
Ungeachtet der geringen Anzahl an konkreten Forschungsaktivitäten an existierenden Strukturen und Anwendungen und den damit verbundenen Einblicken in die Datenerhebung unterschiedlicher \ac{iot}-Systeme, kann basierend auf den bisherigen Resultaten eine Notwendigkeit für weitere Arbeiten auf diesem Gebiet identifiziert werden. Zu wenig ist in den meisten Fällen über fassbares Verhalten der einzelnen Geräte und Anwendungen bekannt, womit neben gesetzlichen Grauzonen auch Risiken für den einzelnen Nutzer im Hinblick auf die Sicherheit seiner Daten entstehen. Dies wird beispielsweise durch die Internationalisierung und den globalen Austausch von Dienstleistungen über territoriale Grenzen hinweg zusätzlich negativ beeinflusst. Aufgrund der unterschiedlichen Auffassungen bezüglich Datenschutz wäre eine globale Vereinheitlichung zur Definition und Klassifikation von schützenswerten Daten und deren Verwendung ein notwendiger Schritt hin zu einer transparenteren Verwendung von \ac{iot}-Applikationen.
Diese sind, nach derzeitigem Stand, entgegen der Erwartungen überwiegend nicht ausreichend dem Nutzer gegenüber in Art und Umfang ihrer enthaltenen Funktionen und deren Auswirkungen auf die Datenerhebung beschrieben. Gleiches gilt auch für smarte Anwendungen in Städten. Ansätze für das Steigern der Achtsamkeit der Bewohner smarter Städte, wie zum Beispiel die "Privacy App" \cite{EUIOTGDPR} sind bereits in Entwicklung, jedoch ist aktuell keine baldige Finalisierung abzusehen, obwohl smarte Systeme bereits eine breite Verwendung innerhalb größerer Städte finden.
Erste Schritte hinsichtlich einer Standardisierung von Systemen für den Bereich \textbf{Smart Home} konnte bereits in Zusammenarbeit unterschiedlicher europäischer Institutionen in Form der \textbf{EN 303 645} \cite{ETSI2020} auf den Weg gebracht werden. Jedoch deckt das Smart Home nur einen Teilbereich der heutigen Einsatzgebiete von \ac{iot} innerhalb des alltäglichen Lebens eines menschlichen Individuums ab. Hier fehlen, Stand der hier vorliegenden Arbeiten, von Seiten des Gesetzgebers allgemeingültige Richtlinien, an denen sich Gerätehersteller, Dienstleister und Nutzer orientieren können.
Zusätzlich ist ein Umdenken in der Entwicklung dieser Systeme unerlässlich. Es hat den Anschein, dass die meisten Gerätehersteller sich keine Gedanken über die Art und den Umfang der Daten machen, die sie erheben. Auch die nachgelagerten Prozesse, wie die Verarbeitung sind teilweise nur lückenhaft dargestellt. Größtenteils werden die Daten in sogenannten Datenseen (auch Data Lakes) unstrukturiert gesammelt und für spätere Auswertungen gelagert \cite{BCG2020}. Dies sollte nicht Sinn und Zweck eines datenzentrierten Dienstes sein, der auf Basis von Echtzeit-Daten dem jeweiligen Verwender optimale Handlungsempfehlungen erstellen soll. Dies würde auch dem Prinzip der \ac{dsgvo} hinsichtlich der \textbf{Datenminimierung} und der \textbf{Speicherbegrenzung} Rechnung tragen. 

%------------------------------------------------------------------------------------------------
\subsection{Ausblick}
\label{sec:Zusammenfassung:ssec:Ausblick}
%------------------------------------------------------------------------------------------------

Nichtsdestotrotz ist mit heutigen Mitteln ein Betreiben von datenschutzkonformen Anwendungen in Kombination mit Technologien des \ac{iot} durchaus möglich. Unter Verwendung von maschinellem Lernen in Verbindung mit Edge-Technologien ist das Betreiben von leistungsfähigerern Applikationen auf den Geräten selbst im Rahmen der derzeit technischen Gegebenheiten realisierbar. 
Somit können pseudonymisierende oder selbst anonymisierende Verfahren auf die erhobenen Daten nach deren Erhebung angewendet werden und würden bereits ab diesem frühen Stadium einen Rückschluss von den gesammelten Daten auf den jeweiligen Betroffenen verhindern. 
Des Weiteren wäre dies ein Wegbereiter für die Option eines Austausches über unterschiedliche Anwendungsdomänen hinweg, um Synergien zwischen einzelnen Anwendungen zu ermöglichen und für die Öffentlichkeit nutzbar zu machen. 
Exemplarisch wäre hier nochmals die Recherche bezüglich des Datenaustausches innerhalb von smarten Städten \cite{BCG2020} zu nennen. Im Zuge der Datenerhebung in 30 unterschiedlichen \textbf{Smart City} Regionen wurden unter Berücksichtigung der Privatsphäre-Regularien öffentliche Datentöpfe geschaffen, die zur Verwendung innerhalb von Hackathons oder anderer Dienstleister zur Verfügung gestellt wurden, um Innovationen zu fördern. 
Beispielgebend fungiert hier die Stadt Louisville in Kentucky, die unter Inanspruchnahme aggregierter Daten aus smarten Asthma-Inhalatoren auf die Luftqualität innerhalb der Stadt schließt. 
Trotz der potentiellen Verbindung durch die Geräte mit einzelnen Personen, lassen sich durch mathematische und technische Mittel Rückschlüsse geschickt vermeiden. 
Konkludierend sind die beiden Pole Regulation und Innovation somit nicht als Antagonisten zu betrachten, sondern ermöglichen durch gegenseitiges Wechselwirken die Entwicklung des jeweils anderen.
