%% ending.tex
%%

%------------------------------------------------------------------------------------------------
\section{Zusammenfassung}
\label{sec:Zusammenfassung}
%------------------------------------------------------------------------------------------------
Ungeachtet der geringen Anzahl an konkreten Forschungsaktivitäten an existierenden Strukturen und Anwendungen und den damit verbundenen Einblicken in die Datenerhebung unterschiedlicher \ac{iot}-Systeme kann basierend auf den bisherigen Resultaten eine Notwendigkeit für weitere Arbeiten auf diesem Gebiet identifiziert werden. Zu wenig ist in den meisten Fällen über fassbares Verhalten der einzelnen Geräte und Anwendungen bekannt, womit neben gesetzlichen Grauzonen auch Risiken für den einzelnen Nutzer im Hinblick auf die Sicherheit seiner Daten entstehen. Dies wird beispielsweise durch die Internationalisierung und den globalen Austausch von Dienstleistungen über territoriale Grenzen hinweg zusätzlich negativ beeinflusst. Aufgrund der unterschiedlichen Auffassungen bezüglich Datenschutz wäre eine globale Vereinheitlichung von rechtlich abgesicherten Daten ein notwendiger Schritt hin zu einer transparenteren Verwendung von \ac{iot}. 
Erste Schritte hinsichtlich einer Standardisierung von Systemen für den Bereich \textbf{Smart Home} konnte bereits in Zusammenarbeit unterschiedlicher europäischer Institutionen in Form der \textbf{EN 303 645} \cite{ETSI2020} auf den Weg gebracht werden. Jedoch deckt das Smart Home nur einen Teilbereich der heutigen Einsatzgebiete von \ac{iot} innerhalb des alltäglichen Lebens eines menschlichen Individuums ab. Hier fehlen, Stand der hier vorliegenden Arbeiten, von Seiten des Gesetzgebers allgemeingültige Richtlinien, an denen sich Gerätehersteller, Dienstleister und Nutzer orientieren können.

%------------------------------------------------------------------------------------------------
\subsection{Ausblick}
\label{sec:Zusammenfassung:ssec:Ausblick}
%------------------------------------------------------------------------------------------------
Nichtsdestotrotz ist mit heutigen Mitteln ein Betreiben von Anwendungen in Kombination mit \ac{iot} durchaus möglich. Die Kombination von maschinellem Lernen in Verbindung mit Edge-Technologien ebnen den Weg für die Nutzung leistungsfähigerer Applikationen auf den Geräten selbst, die somit pseudonymisierende oder selbst anonymisierende Verfahren auf die erhobenen Daten anwenden können. Dies würde bereits ab dem Punkt der Erhebung einen Rückschluss von Daten auf den jeweiligen Betroffenen verhindern und die Informationen für unterschiedliche Anwendungszwecke und den Austausch zwischen einzelnen Bereichen für ein gegenseitiges Profitieren nutzbar machen. Exemplarisch wäre hier nochmals die Recherche bezüglich des Datenaustausches innerhalb von smarten Städten \cite{BCG2020} zu nennen. Im Zuge der Datenerhebung in 30 unterschiedlichen \textbf{Smart City} Regionen wurden unter Berücksichtigung der Privatsphäre-Regularien öffentliche Datentöpfe geschaffen, die zur Verwendung innerhalb von Hackathons oder anderer Dienstleister zur Verfügung gestellt wurden, um Innovationen zu fördern. Beispielgebend fungiert hier die Stadt Louisville in Kentucky, die unter Inanspruchnahme aggregierter Daten aus smarten Asthma-Inhalatoren auf die Luftqualität innerhalb der Stadt schließt. Trotz der potentiellen Verbindung durch die Geräte mit einzelnen Personen, lassen sich durch mathematische und technische Mittel Rückschlüsse geschickt vermeiden. Aus diesem Grund sollte auch beachtet werden, dass die Regulation nicht zu einem Dämpfer der Innovation werden sollte.
