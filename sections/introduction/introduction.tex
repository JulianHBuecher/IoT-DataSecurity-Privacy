%% introduction.tex
%%

%------------------------------------------------------------------------------------------------
\section{Einleitung}
\label{sec:Einleitung}
%------------------------------------------------------------------------------------------------

Im Rahmen dieses Kapitels wird zu Beginn im Rahmen der Motivation die Notwendigkeit für die Bearbeitung dieser Thematik hergeleitet. Darauf aufbauend wird im anschließenden Grundlagenteil, die benötigte Nomenklatur und das Verständnis für die Konzepte innerhalb dieses Themenbereichs gelegt. Abschließend erfolgt basierend auf den gesammelten Informationen der Übergang hin zum Hauptteil der Ausarbeitung.

%------------------------------------------------------------------------------------------------
\subsection{Motivation}
\label{sec:Einleitung:ssec:Motivation}
%------------------------------------------------------------------------------------------------

Das \ac{iot} und dessen Anwendungen nehmen im Rahmen der immer weiter voranschreitenden Entwicklungen im Bereich kleinster Sensoren und hochleistungsfähiger Chips eine immer wichtigere Rolle in der Kontrolle und Steuerung komplexer Systeme ein. 
Ihr Einsatz reicht von der Implementierung in hochautomatisierten Fertigungsstraßen bis hin zu kleinsten medizinischen Geräten, die innerhalb oder als Peripherie durch den Menschen genutzt werden können. Über die Sensoren können somit gezielt Daten aus deren Umfeld erhoben und für spätere Optimierungen analysiert werden. Im industriellen Bereich wären dies zum Beispiel die Möglichkeiten zur gezielteren Wartung von einzelnen Fertigungsanlagen, auch als Predictive Maintenance bekannt, oder der fein-granularen Überwachung von einzelnen Produktionsschritten. 
Aufgrund ihrer Allgegenwärtigkeit, auch innerhalb des privaten Bereichs, sammeln diese Geräte neben rein technischen auch unweigerlich Daten, die im Zusammenhang mit menschlichen Nutzern stehen. Diese sogenannten \ac{pii} sind eine besondere Kategorie von Informationen, die es ermöglichen Rückschlüsse auf den jeweiligen Nutzer zu ziehen. Aus diesem Grunde sind sie, insbesondere in der heutigen Zeit, besonders schützenswert.
Genau aus diesem Grund entstanden rechtliche Rahmenbedingungen, wie zum Beispiel die \ac{gdpr} auf europäischer Ebene \cite{dsgvo2016}, die die Handhabung solcher Daten reguliert. Jedoch liegt die Einhaltung dieser Formalitäten stets bei den Herstellern von Smart Devices und deren Applikationen. Eine Überprüfung der Umsetzung innerhalb der einzelnen Geräte und ihrer Applikationen liegt im Vertrauen des Nutzers und muss für jedes Gerät im Einzelfall geprüft werden.
