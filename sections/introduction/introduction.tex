%% introduction.tex
%%

%------------------------------------------------------------------------------------------------
\section{Einleitung}
\label{sec:Einleitung}
%------------------------------------------------------------------------------------------------

%Im Rahmen dieses Kapitels wird zu Beginn im Rahmen der Motivation die Notwendigkeit für die Bearbeitung dieser Thematik hergeleitet. Darauf aufbauend wird im anschließenden Grundlagenteil, die benötigte Nomenklatur und das Verständnis für die Konzepte innerhalb dieses Themenbereichs gelegt. Abschließend erfolgt basierend auf den gesammelten Informationen der Übergang hin zum Hauptteil der Ausarbeitung.

%%------------------------------------------------------------------------------------------------
%\subsection{Motivation}
%\label{sec:Einleitung:ssec:Motivation}
%%------------------------------------------------------------------------------------------------

Das \ac{iot} und dessen Anwendungen nehmen im Rahmen der kontinuierlich voranschreitenden Digitalisierung und der Entwicklungen im Bereich kleinster Sensoren und hochleistungsfähiger Chips eine immer wichtigere Rolle in der Kontrolle und Steuerung komplexer Systeme ein.
Der Einsatz reicht von der Implementierung hochautomatisierter Fertigungsstraßen bis hin zu kleinsten medizinischen Geräten, die als Teil kurativer oder präventiver Behandlungsmethoden durch den Menschen genutzt werden können. Über die Sensoren können somit gezielt Daten aus deren Umfeld erhoben und als Teil von Analysen für die spätere Optimierungen genutzt werden. Im industriellen Bereich wären dies zum Beispiel die Möglichkeiten zur gezielteren Wartung von einzelnen Fertigungsanlagen, auch Predictive Maintenance genannt, oder der fein-granularen Überwachung von einzelnen Produktionsschritten.
Aufgrund ihrer Allgegenwärtigkeit, auch innerhalb des privaten Bereichs, sammeln diese Geräte neben rein technischen auch unweigerlich Daten, die im Zusammenhang mit menschlichen Nutzern stehen. Diese sogenannten \ac{pii} oder personenbezogenen Daten sind eine besondere Kategorie von Informationen, die es ermöglichen direkt oder indirekt Rückschlüsse auf den jeweiligen Nutzer zu ziehen. Aus diesem Grunde sind sie, insbesondere in der heutigen Zeit, besonders schützenswert und unterliegen höchsten Sicherheitsanforderungen.
Genau aus diesem Grund entstanden rechtliche Rahmenbedingungen, wie zum Beispiel die \ac{gdpr} auf europäischer Ebene \cite{dsgvo2016}, die die Handhabung solcher Daten reguliert.

%%------------------------------------------------------------------------------------------------
%\subsection{Wissenschaftliche Fragen}
%\label{sec:Einleitung:ssec:Wissenschaftliche Fragen}
%%------------------------------------------------------------------------------------------------

\noindent Die Kontrolle der Einhaltung dieser Vorgaben obliegt hierbei meistens den Nutzern selbst oder gemeinnützigen Institutionen, die gezielt einzelne Geräte identifizieren und auf deren Konformität hin testen. Dieser Aspekt wird im Rahmen dieser Arbeit betrachtet und untersucht auf Basis vorhandener Analysen von Geräte und deren Applikationen aus dem Bereich Smart Home und Smart City und deren Einhaltung von Datenschutz. Bezüglich dieser allgemeinen Fragestellung werden die nachfolgenden \ac{rq} definiert, die basierend auf den Ergebnissen dieser Arbeit beantwortet werden sollen.\\
\textbf{RQ1.} Welche (Nutzer-)Daten werden von \ac{iot}-Geräten (bewusst) gesammelt? Hiermit werden basierend auf den bereitgestellten Diensten die erforderlichen Daten analysiert, die für die Erbringung des Dienstes notwendig sind und vorhandene Ausnahmen identifiziert, die nicht unmittelbar mit dem Dienst in Verbindung stehen.\\
\textbf{RQ2.} Was passiert mit den Daten und wer verwendet sie? Aufbauend auf der ersten Forschungsfrage wird nun betrachtet, wie das Gerät die nun vorhandenen Daten verwaltet beziehungsweise prozessiert. Hierbei wird speziell betrachtet, ob die Daten das Gerät in Form von Anfragen an externe Dienste zur Bereitstellung eines Dienstes verlassen oder sie im Rahmen von periodischen Wartungs- oder Überwachungsaufgaben exponiert werden.\\
\textbf{RQ3}. Hat der Nutzer Einfluss darauf, welche Daten von ihm gesammelt werden? Dabei kommt zum Tragen, ob das Gerät eine Möglichkeit zur Konfiguration bietet, die entweder Teil einer für den Nutzer zugänglichen Bedienmöglichkeit, beispielsweise einer Applikation für das Smartphone oder einem Web-Interface, bietet oder auf Basis von kontextuellen Rahmenbedingungen, wie zum Beispiel seinem aktuellen Standort, seine Tätigkeiten im Bezug auf das Sammeln von Daten anpasst.\\
\textbf{RQ4.} Wie haben Regularien, unter anderem \ac{dsgvo} oder \ac{ftc}, Einfluss auf das Verhalten von \ac{iot}-Geräten? Indirekt durch RQ3 bereits angedeutet, wird mittels RQ4 die kontextuellen Rahmenbedingungen expliziter in Form von gesetzlichen Rahmenbedingungen gefasst.\\
\textbf{RQ5.} Kann unkontrolliertes Datensammeln (technisch) eingeschränkt werden? Basierend auf den vorangegangenen Untersuchungen und deren Ergebnisse hinsichtlich der Erhebung von Daten der unterschiedlichen Geräte, stellt sich nun die Frage, inwiefern dieses Verhalten durch technische oder gesetzliche Möglichkeiten reguliert werden kann.\\
\textbf{RQ6.} Gibt es Grenzen für den Einsatz von \ac{iot} (beispielsweise im öffentlichen Bereich)? Unter Berücksichtigung der gesetzlich vorgeschriebenen Konventionen für den Einsatz und den Betrieb von \ac{iot}-Geräten und deren Anwendungen soll nun abschließend geklärt werden, ob die Anwendung von \ac{iot} auch Grenzen besitzt, die im aktuellen Kontext nicht ohne weiteres überwunden werden können. \\

%%------------------------------------------------------------------------------------------------
%\subsection{Struktur der Arbeit}
%\label{sec:Einleitung:ssec:Struktur der Arbeit}
%%------------------------------------------------------------------------------------------------

\noindent Nachfolgend wird nun die Struktur der vorliegenden Arbeit anhand der einzelnen Sektionen kurz erläutert. Die Sektion \fullref{sec:Einleitung} enthält neben einer allgemeinen Einführung und der Motivation zur Bearbeitung der behandelten Thematiken die relevanten Forschungsfragen (RQ) ein, die auf Basis der gesammelten Ergebnisse beantwortet werden sollen. Innerhalb von \fullref{sec:Grundlagen} enthält neben den Definitionen fundamentaler Begrifflichkeiten aus dem Kontext des \ac{iot} und den Aufbau ihrer Systeme eine Einführung in die derzeit geltenden Regelwerke. Hierbei werden die unterschiedlichen Kategorien von Daten betrachtet und die notwendigen Stellen innerhalb der gesetzlichen Ausarbeitungen identifiziert, die für eine Bewertung der Ergebnisse aus den späteren Analysen unabdingbar sind. Außerdem werden potentielle Risiken gelistet, die infolge von falscher Handhabung oder nicht rechtskonformer Verarbeitung für den Nutzer entstehen können. Das anschließende Sektion \fullref{sec:Analyse der Datenschutz-Konformität} stellt die Ergebnisse der recherchierten Analysen und Grundlagenarbeiten zu dem behandelten Thema vor. Neben der Auflistung der gesammelten Resultate werden darauf aufbauend Regulationsmöglichkeiten technischer, wie auch nicht technischer Art aufgezeigt, die das Verhalten der betrachteten Geräte beziehungsweise Kontexte reguliert. Schließend erfolgt mit der Sektion \fullref{sec:Zusammenfassung} eine Aggregation und Interpretation der Synthesen aus den vorangegangenen Sektionen und ein Ausblick hinsichtlich weiterer Forschungsmöglichkeiten und potentieller Optimierungsmöglichkeiten der aktuellen Situation.

