%% introduction.tex
%%

%------------------------------------------------------------------------------------------------
\section{Einleitung}
\label{sec:Einleitung}
%------------------------------------------------------------------------------------------------

Das \ac{iot} und dessen Anwendungen nehmen im Rahmen der kontinuierlich voranschreitenden Digitalisierung und der Entwicklungen im Bereich kleinster Sensoren und hochleistungsfähiger Chips eine immer wichtigere Rolle in der Kontrolle und Steuerung komplexer Systeme ein.
Der Einsatz reicht von der Implementierung hochautomatisierter Fertigungsstraßen bis hin zu kleinsten medizinischen Geräten, die als Teil kurativer oder präventiver Behandlungsmethoden durch den Menschen genutzt werden können.
Über die Sensoren können somit gezielt Daten aus deren Umfeld erhoben und als Teil von Analysen für spätere Optimierungen genutzt werden. Im industriellen Bereich wären dies zum Beispiel die Möglichkeiten zur gezielteren Wartung von einzelnen Anlagen im Zusammenhang mit Industrie 4.0, auch vorausschauende Wartung genannt, oder der fein-granularen Überwachung von einzelnen Produktionsschritten.
Aufgrund ihrer Allgegenwärtigkeit, auch innerhalb des privaten Bereichs, sammeln diese Geräte neben rein technischen auch unweigerlich Daten, die im Zusammenhang mit menschlichen Nutzern stehen.
Diese sogenannten \ac{pii}, wie sie im Kontext amerikanischer Definitionen geführt werden, oder personenbezogenen Daten, entsprechend der Definition aus dem europäischen Raum, sind eine besondere Kategorie von Informationen, die es ermöglichen direkt oder indirekt Rückschlüsse auf den jeweiligen Nutzer zu ziehen. Aus diesem Grunde sind sie, insbesondere in der heutigen Zeit, besonders schützenswert und unterliegen höchsten Sicherheitsanforderungen.
Genau aus diesem Grund entstanden rechtliche Rahmenbedingungen, wie zum Beispiel die \ac{gdpr} auf europäischer Ebene \cite{Dsgvo2016}, die die Handhabung solcher Daten reguliert.

%%------------------------------------------------------------------------------------------------
% Auflistung der zu diskutierenden Aspekte
%%------------------------------------------------------------------------------------------------

\noindent Die Kontrolle der Einhaltung dieser Vorgaben obliegt hierbei meistens den Nutzern selbst oder domänenspezifischen Institutionen, die gezielt einzelne Geräte identifizieren und auf deren Konformität hin 
Diese Problematik wird im Rahmen der hier vorliegenden Arbeit betrachtet und auf Basis vorhandener Analysen unterschiedlicher Geräte und deren Applikationen aus dem Bereich Smart Home und Smart City bezüglich deren Einhaltung geltender Datenschutzbestimmungen ausgewertet. Bezüglich einer besseren Darstellung und Einordnung der einzelnen zu diskutierenden \ac{asp}, werden diese in Form von Fragen nachfolgend aufgeführt.
Die Formulierung der Fragen orientiert sich hierbei an dem Aufbau der hier vorliegenden Arbeit und betrachtet dementsprechend zuerst die aktuelle Situation, wie sie in den einzelnen Domänen zu finden ist und geht über in eine genauere Betrachtung hinsichtlich einzelner Aspekte zur Wahrung des Datenschutzes. Somit wird ein kontextueller Rahmen aufgespannt, der entsprechend des Umfangs der Arbeit eine dennoch detaillierte Betrachtung der einzelnen Teilaspekte ermöglicht.\\
\textbf{Asp1.} Welche (Nutzer-)Daten werden von \ac{iot}-Geräten (bewusst) gesammelt? Hiermit werden basierend auf den bereitgestellten Diensten die erforderlichen Daten analysiert, die für die Erbringung des Dienstes notwendig sind und vorhandene Ausnahmen identifiziert, die nicht unmittelbar mit dem Dienst in Verbindung stehen.\\
\textbf{Asp2.} Was passiert mit den Daten und wer verwendet sie? Aufbauend auf der ersten Forschungsfrage wird nun betrachtet, wie das Gerät die nun vorhandenen Daten verwaltet beziehungsweise prozessiert. Hierbei wird speziell betrachtet, ob die Daten das Gerät in Form von Anfragen an externe Dienste zur Bereitstellung eines Dienstes verlassen oder sie im Rahmen von periodischen Wartungs- oder Überwachungsaufgaben exponiert werden.\\
\textbf{Asp3}. Hat der Nutzer Einfluss darauf, welche Daten von ihm gesammelt werden? Dabei kommt zum Tragen, ob das Gerät eine Möglichkeit zur Konfiguration bietet, die entweder Teil einer für den Nutzer zugänglichen Bedienmöglichkeit, beispielsweise einer Applikation für das Smartphone oder einem Web-Interface, bietet oder auf Basis von kontextuellen Rahmenbedingungen, wie zum Beispiel seinem aktuellen Standort, seine Tätigkeiten im Bezug auf das Sammeln von Daten anpasst.\\
\textbf{Asp4.} Wie haben Regularien, unter anderem \ac{dsgvo} oder \ac{ftc}, Einfluss auf das Verhalten von \ac{iot}-Geräten? Indirekt durch RQ3 bereits adressiert, wird mittels RQ4 die kontextuellen Restriktionen expliziter in Form von gesetzlichen Rahmenbedingungen gefasst.\\
\textbf{Asp5.} Kann unkontrolliertes Datensammeln (technisch) eingeschränkt werden? Basierend auf den vorangegangenen Untersuchungen und deren Ergebnisse hinsichtlich der Erhebung von Daten der unterschiedlichen Geräte, stellt sich nun die Frage, inwiefern dieses Verhalten durch technische oder gesetzliche Möglichkeiten reguliert werden kann.\\
\textbf{Asp6.} Gibt es Grenzen für den Einsatz von \ac{iot} (beispielsweise im öffentlichen Bereich)? Unter Berücksichtigung der gesetzlich vorgeschriebenen Konventionen für den Einsatz und den Betrieb von \ac{iot}-Geräten und deren Anwendungen soll nun abschließend geklärt werden, ob die Anwendung von \ac{iot} auch Grenzen besitzt, die im aktuellen Kontext nicht ohne weiteres überwunden werden können.

%%------------------------------------------------------------------------------------------------
% Struktur der Arbeit
%%------------------------------------------------------------------------------------------------

\noindent Nachfolgend wird nun die Struktur und der Inhalt der einzelnen Sektionen kurz erläutert. 
Zu Beginn der Arbeit wird in Sektion \fullref{sec:Einleitung} eine allgemeine Hinführung und Motivation hinsichtlich der zu behandelnden Thematik durchgeführt. Die hieraus abgeleiteten \ac{asp}, die im Kontext der Ausarbeitung diskutiert werden sollen, bilden anschließend den Übergang hin zur nächsten Sektion. 
Innerhalb von \fullref{sec:Grundlagen} sind neben den Definitionen fundamentaler Begrifflichkeiten aus dem Kontext des \ac{iot} und der Aufbau beteiligter Systeme, eine Einführung in die derzeit geltenden Regelwerke integriert. 
Hierbei werden die unterschiedlichen Kategorien von Daten spezifiziert und die notwendigen Stellen innerhalb der gesetzlichen Ausarbeitungen identifiziert, die für eine Bewertung der Ergebnisse aus den späteren Analysen unabdingbar sind. 
Außerdem werden potentielle Risiken gelistet, die infolge von falscher Handhabung oder nicht rechtskonformer Verarbeitung für den Nutzer entstehen können. Die anschließende Sektion \fullref{sec:Analyse der Datenerhebung} stellt die Ergebnisse der recherchierten Analysen und Grundlagenarbeiten zu dem behandelten Thema vor. 
Neben der Auflistung der gesammelten Resultate werden darauf aufbauend Regulationsmöglichkeiten technischer, wie auch konzeptioneller Art vorgestellt, die das Verhalten der betrachteten Geräte beziehungsweise Kontexte regulieren können. 
Abschließend erfolgt mit der Sektion \fullref{sec:Zusammenfassung} eine Aggregation und Interpretation der Schlussfolgerungen aus den vorangegangenen Sektionen und ein Ausblick hinsichtlich weiterer Forschungsmöglichkeiten und potentieller Optimierungsmöglichkeiten der aktuellen Situation. 

