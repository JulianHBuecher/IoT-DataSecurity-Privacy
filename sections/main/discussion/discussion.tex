
%------------------------------------------------------------------------------------------------
\section{Diskussion der \acl{asp}}
\label{sec:Hauptteil:ssec:Diskussion der Aspekte}
%------------------------------------------------------------------------------------------------

In Anbetracht der konkludierten Ergebnisse hinsichtlich der Literaturrecherche werden anschließend die einzelnen zu Beginn dieser Arbeit abgeleiteten \acl{asp} in Bezug ihrer Fragestellungen nochmals aufgeführt und entsprechend der Resultate aus den Analysen diskutiert. Bezüglich \textbf{\ac{asp}1} wurde das bewusste, wie auch unbewusste Erfassen von Daten fokussiert. Resultierend konnte festgestellt werden, dass sowohl Geräte aus dem Bereich \textbf{Smart Home}, wie auch der \textbf{Smart City} die Prädisposition besitzen, neben rein technischen auch personenbezogene Daten ohne Kenntnis beteiligter Individuen zu sammeln, die sich in ihrem Wirkungsbereich aufhalten. 
Hinsichtlich der Verwendung dieser Daten, adressiert von \textbf{\ac{asp}2}, steht deren nachgelagerte Prozessierung im im Fokus dieses \aclp{asp}. Diesbezüglich muss differenziert werden, ob die Geräte selbst fähig sind, auf Basis dieser Informationen ihre Dienste bereitzustellen oder ein Service außerhalb des Gerätes für dessen Verarbeitung zuständig ist. Generalisierend sind die meisten Geräte mit Cloud-Diensten gekoppelt, die aufgrund der begrenzten Rechen- und Speicherkapazitäten von \ac{iot}-Geräten die aufwändige Verarbeitung übernehmen. Nur in einigen Ausnahmefällen (siehe ''SmartWalk'' \cite{Natix2022}) wird vor dem Versenden auf einem Edge-Layer eine Präprozessierung durchgeführt, um datenschutzkonform personenbezogene Daten vorab anonymisieren. Dieses Vorgehen ist jedoch nicht pauschal im Einsatz. 
Zum Thema der Konfigurierbarkeit der Datenerhebung basierend auf angebotenen Schnittstellen der Geräte, die in \textbf{\ac{asp}3} evaluiert wird, lässt sich bezüglich der beschriebenen Regulationsmöglichkeiten schlussfolgern, dass die Anwendungen und Systeme in den betrachteten Fällen nicht durch den Nutzer über Anpassungsmöglichkeiten im Hinblick auf ihre Aktivität des Datensammelns einzuschränken sind. 
Speziell die beiden Arbeiten \cite{Mandalari2021,Ren2019} haben gezeigt, dass wie in \textbf{\ac{asp}4} kritisch beleuchetet, die unterschiedlichen rechtlichen Rahmenwerke durchaus einen Einfluss auf das Verhalten der betrachteten \ac{iot}-Geräte hat. Rückblickend konnte unter anderem festgestellt werden, dass Geräte aus den USA deutlich mehr Daten an nicht dienstbezogene Destinationen senden, wie es bei Geräten in dem europäischen Versuchslabor der Fall war. Dies konnte exemplarisch durch die Verwendung eines VPN-Tunnels nachweislich verifiziert werden. 
Entsprechend der beleuchteten Werkzeuge für die Einschränkung beziehungsweise Herstellung einer datenschutzkonformen Erhebung von Daten ist die Umsetzung dieser Formalia durchaus möglich. Neben technischen Mitteln, wie sie im Rahmen dieser Arbeit beschrieben wurden, sind auch konzeptionelle Ansätze wie beispielsweise das \ac{sbd} und \ac{pbd} wichtige Ansätze, die bereits bei der Entwicklung von Systemen und deren Anwendungen die Perspektive eröffnen einen Bruch geltender Rechtssprechungen zu vermeiden. Dies lässt eine (technische) Regulation des Datensammelns im Kontext von \textbf{\ac{asp}5} durchaus zu. 
Entsprechend einer ganzheitlichen Betrachtung vorherigen Fragen ist eine Beantwortung von \textbf{\ac{asp}6} hinsichtlich Grenzen für den Einsatz von \ac{iot} dahingehend möglich, dass unter Berücksichtigung geltender Datenschutzbestimmungen und der Verwendung innovativer Technologien der Einsatz insoweit viabel ist, wie es die Grundrechte einer einzelnen Person nicht verletzt. Als Negativbeispiel wäre der Einsatz von smarten Technologien zur gezielten Detektion von Personengruppen, wie sie im Falle der Uighuren in Xinjiang und Kashgar \cite{Drexel2020} durch die Installation von smarten Kamerasystemen vorgefallen ist. Gestützt durch maschinelle Lernmodelle wurde hiermit die Identifikation von Personen dieser ethnischen Gruppe ermöglicht und deren Verfolgung realisiert. 
Können diese Anwendungen durch die bereits genannten Standards oder eine Ausweitung von gesetzlichen Rahmenbedingungen reguliert werden, so steht einer Integration von \ac{iot}-Technologien in weitere Bereiche vorerst nichts im Wege. Jedoch gilt es auch hier anzumerken, dass diese Aussage keineswegs zu generalisieren ist und im Einzelfall separat betrachtet werden sollte. 
