
%------------------------------------------------------------------------------------------------
\section{Diskussion der \acl{asp}}
\label{sec:Hauptteil:ssec:Diskussion der Aspekte}
%------------------------------------------------------------------------------------------------

In Anbetracht der aggregierten Ergebnisse hinsichtlich der Literaturrecherche werden anschließend die einzelnen zu Beginn dieser Arbeit abgeleiteten \ac{asp} in Bezug ihrer Fragestellungen nochmals aufgeführt und entsprechend der Resultate aus den Analysen diskutiert. Bezüglich \textbf{\ac{asp}1} wurde das bewusste, wie auch unbewusste Erfassen von Daten fokussiert. Resultierend konnte festgestellt werden, dass sowohl Geräte aus dem Bereich \textbf{Smart Home}, wie auch der \textbf{Smart City} die Prädisposition besitzen, neben rein technischen auch personenbezogene Daten ohne Kenntnis beteiligter Individuen zu sammeln, die sich in ihrem Wirkungsbereich aufhalten.\\
Hinsichtlich der Verwendung dieser Daten wird mittels \textbf{\ac{asp}2} adressiert, was mit den erhobenen Daten im Nachhinein passiert. Diesbezüglich muss differenziert werden, ob die Geräte selbst fähig sind, auf Basis dieser Informationen ihre Dienste bereitzustellen oder ein Service außerhalb des Gerätes für dessen Verarbeitung zuständig ist. Generalisierend sind die meisten Geräte mit Cloud-Diensten gekoppelt, die aufgrund der begrenzten Rechen- und Speicherkapazitäten von \ac{iot}-Geräten die aufwändige Verarbeitung übernehmen. Nur in einigen Ausnahmefällen (siehe "SmartWalk" \cite{Natix2022}) wird vor dem Versenden auf einem Edge-Layer eine Präprozessierung durchgeführt, um datenschutzkonform personenbezogene Daten zu anonymisieren. Dieses Vorgehen ist jedoch nicht pauschal im Einsatz.\\
Zum Thema der Konfigurierbarkeit der Datenerhebung basierend auf angebotenen Schnittstellen der Geräte, die in \textbf{\ac{asp}3} angesprochen wird, lässt sich bezüglich der beschriebenen Regulationsmöglichkeiten schlussfolgern, dass die Anwendungen und Systeme in den betrachteten Fällen nicht durch den Nutzer über Anpassungsmöglichkeiten im Hinblick auf ihre Aktivität des Datensammelns einzuschränken sind.\\
Speziell die beiden Arbeiten \cite{Mandalari2021,Ren2019} haben gezeigt, dass wie in \textbf{\ac{asp}4} angemerkt, die unterschiedlichen rechtlichen Rahmenwerke durchaus einen Einfluss auf das Verhalten der betrachteten \ac{iot}-Geräte hat. Rückblickend konnte beispielsweise festgestellt werden, dass Geräte aus den USA deutlich mehr Daten an nicht dienstbezogene Destinationen senden, wie es bei Geräten in dem europäischen Versuchslabor der Fall war. Dies konnte exemplarisch durch die Verwendung eines VPN-Tunnels nachweislich nochmals verifiziert werden.\\
Entsprechend der beleuchteten Werkzeuge für die Einschränkung beziehungsweise Herstellung einer datenschutzkonformen Erhebung von Daten durchaus möglich. Neben technischen Mitteln, wie sie im Rahmen dieser Arbeit beschrieben wurden, sind auch konzeptionelle Ansätze wie beispielsweise das \ac{sbd} und \ac{pbd} wichtige Ansätze, die bereits bei der Entwicklung von Systemen und deren Anwendungen die Perspektive eröffnen einen Bruch geltender Rechtssprechungen zu vermeiden. Dies lässt eine (technische) Regulation des Datensammelns im Kontext von \textbf{\ac{asp}5} durchaus zu.\\
Durch ein Konkludieren der vorherigen Fragen ist eine Beantwortung von \textbf{\ac{asp}6} hinsichtlich Grenzen für den Einsatz von \ac{iot} dahingehend möglich, dass unter Berücksichtigung geltender Datenschutzbestimmungen und der Verwendung innovativer Technologien der Einsatz soweit möglich ist, wie es die Grundrechte einer einzelnen Person nicht einschränkt. Vergleichsweise wäre der Einsatz von smarten Technologien zur gezielten Verfolgung von Personengruppen, wie sie im Falle der Uighuren in Xinjiang und Kashgar \cite{Drexel2020} durch die Installation von smarten Kamerasystemen erfolgte, die gestützt durch maschinelle Lernmodelle die Identifikation von Personen dieser ethnischen Gruppe ermöglichen.
