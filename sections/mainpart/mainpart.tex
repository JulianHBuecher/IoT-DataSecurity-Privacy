%% mainpart.tex
%%

%------------------------------------------------------------------------------------------------
\section{Analyse der Datenschutz-Konformität}
\label{sec:Analyse der Datenschutz-Konformität}
%------------------------------------------------------------------------------------------------

Die Struktur des nachfolgenden Kapitels umfasst neben der Analyse der erhobenen Daten aus den Kontexten \textbf{Smart City} und \textbf{Smart Home} die kritische Auseinandersetzung hinsichtlich ihrer Konformität im Rahmen der \ac{dsgvo}, die hier gewählte rechtliche Rahmenbedingung. Darauf aufbauend werden resultierend auf den Ergebnissen Handlungsempfehlungen und Möglichkeiten aufgezeigt, die es den Nutzern oder \ac{ds} ermöglichen sollen, nach eigenem Ermessen die Erhebung der von ihren Geräten gesammelten Daten zu regulieren. Abschließend erfolgt eine Bewertung der gesammelten Ergebnisse.

%------------------------------------------------------------------------------------------------
\subsection{Datenerhebung}
\label{sec:Hauptteil:ssec:Datenerhebung}
%------------------------------------------------------------------------------------------------

Beginnend mit der Analyse der Datenerhebung anhand der einzelnen Geräte in den vorliegenden Kontexten. Die Wahl von \textbf{Smart City} und \textbf{Smart Home} als Gegenstand der Analyse wurde aufgrund der hohen Interaktionsrate zwischen den einzelnen Geräten und den \textbf{ds} gewählt. Neben Smart TVs, Smart Watches und Sensoren interagieren die Nutzer fast minütlich mit den entsprechenden Geräten.

%------------------------------------------------------------------------------------------------
\subsubsection{Smart Home}
\label{sec:Hauptteil:ssec:Datenerhebung:sssec:Smart Home}
%------------------------------------------------------------------------------------------------

Als Basis für die nachfolgenden Untersuchungen werden die Veröffentlichungen  \cite{Mandalari2021,Ren2019} als Ausgangspunkte herangezogen. Fokus der beiden Ausarbeitungen ist die Analyse des Datenverkehrs ausgehend von \textbf{Smart Devices} im \textbf{Smart Home} Umfeld mit einer anschließenden Regulation des Informationsflusses an nicht dienstbezogenen Endpunkten. 

%------------------------------------------------------------------------------------------------
\subsubsection{Smart City}
\label{sec:Hauptteil:ssec:Datenerhebung:sssec:Smart City}
%------------------------------------------------------------------------------------------------

%------------------------------------------------------------------------------------------------
\subsection{Datenschutzkonformität}
\label{sec:Hauptteil:ssec:Datenschutzkonformität}
%------------------------------------------------------------------------------------------------

%------------------------------------------------------------------------------------------------
\subsection{Regulationsmöglichkeiten}
\label{sec:Hauptteil:ssec:Regulationsmöglichkeiten}
%------------------------------------------------------------------------------------------------

%------------------------------------------------------------------------------------------------
\subsubsection{IoTrimmer \& IoTrigger}
\label{sec:Hauptteil:ssec:Regulationsmöglichkeiten:sssec:IoTrimmer und IoTrigger}
%------------------------------------------------------------------------------------------------

%------------------------------------------------------------------------------------------------
\subsubsection{IoTInspector}
\label{sec:Hauptteil:ssec:Regulationsmöglichkeiten:sssec:IoTInspector}
%------------------------------------------------------------------------------------------------

%------------------------------------------------------------------------------------------------
\subsection{Bewertung der Ergebnisse}
\label{sec:Hauptteil:ssec:Bewertung der Ergebnisse}
%------------------------------------------------------------------------------------------------

